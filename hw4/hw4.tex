\documentclass{tufte-handout}

\title{Math 451: Homework 4}
\author{Anthony Brice}

\usepackage{graphicx} % allow embedded images
\setkeys{Gin}{width=\linewidth,totalheight=\textheight,keepaspectratio}
% \graphicspath{{graphics/}} % set of paths to search for images
\usepackage{amsmath, amsthm, amssymb, amsfonts}  % extended mathematics
\usepackage{IEEEtrantools} % for fancy equations
\usepackage{booktabs} % book-quality tables
\usepackage{units}    % non-stacked fractions and better unit spacing
\usepackage{multicol} % multiple column layout facilities
\usepackage{multirow} % ???
\usepackage{lipsum}   % filler text
\usepackage{fancyvrb} % extended verbatim environments
  \fvset{fontsize=\normalsize}% default font size for fancy-verbatim
                              % environments
\usepackage[inline]{enumitem} % for fancy lists

\usepackage{wasysym} % for that creepy blank-expression smiley

\usepackage[T1]{fontenc}
\usepackage{ccfonts}
\usepackage[euler-digits,euler-hat-accent]{eulervm}
\usepackage{microtype}

\usepackage{newfloat}
\DeclareFloatingEnvironment[name=Listing]{mylisting}

\newenvironment{listingenv} {
  \begin{mylisting}
} {
  \end{mylisting}
}

\usepackage{subcaption}
\captionsetup{compatibility=false}

% \usepackage[scaled]{berasans}
% \usepackage[T1]{fontenc}
% \usepackage{listings,xcolor}
% %\lstloadlanguages{[5.2]Mathematica}
% \lstset{language=Mathematica}

% \lstset{basicstyle={\sffamily\footnotesize},
%   numbers=left,
%   numberstyle=\tiny\color{gray},
%   numbersep=5pt,
%   breaklines=true,
%   captionpos={t},
%   frame={lines},
%   rulecolor=\color{black},
%   framerule=0.5pt,
%   columns=flexible,
%   tabsize=2
% }

% Standardize command font styles and environments
\newcommand{\doccmd}[1]{\texttt{\textbackslash#1}}% command name -- adds backslash automatically
\newcommand{\docopt}[1]{\ensuremath{\langle}\textrm{\textit{#1}}\ensuremath{\rangle}}% optional command argument
\newcommand{\docarg}[1]{\textrm{\textit{#1}}}% (required) command argument
\newcommand{\docenv}[1]{\textsf{#1}}% environment name
\newcommand{\docpkg}[1]{\texttt{#1}}% package name
\newcommand{\doccls}[1]{\texttt{#1}}% document class name
\newcommand{\docclsopt}[1]{\texttt{#1}}% document class option name
\newenvironment{docspec}{\begin{quote}\noindent}{\end{quote}}
% command specification environment

\newcommand{\e}[1]{\ensuremath{\times 10^{#1}}} % Macro for scientific
                                                % notation

\DeclareMathOperator{\Log}{Log}
\DeclareMathOperator{\Arg}{Arg}
\let\Im\relax
\DeclareMathOperator{\Im}{Im}
\let\Re\relax
\DeclareMathOperator{\Re}{Re}
\DeclareMathOperator{\sech}{sech}
\DeclareMathOperator{\csch}{csch}
\DeclareMathOperator{\arcsec}{arcsec}
\DeclareMathOperator{\arccot}{arccot}
\DeclareMathOperator{\arccsc}{arccsc}
\DeclareMathOperator{\arccosh}{arccosh}
\DeclareMathOperator{\arcsinh}{arcsinh}
\DeclareMathOperator{\arctanh}{arctanh}
\DeclareMathOperator{\arcsech}{arcsech}
\DeclareMathOperator{\arccsch}{arccsch}
\DeclareMathOperator{\arccoth}{arccoth}

% Use fancy symbols for footnotes
\usepackage{hyperref}
\usepackage{natbib}
\renewcommand{\thefootnote}{\fnsymbol{footnote}}%
\usepackage{perpage}
\MakePerPage{footnote}

\usepackage{manfnt} % ???

\usepackage{mathtools} % for \DeclarePairedDelimiter

\DeclarePairedDelimiter\abs{\lvert}{\rvert}%
\DeclarePairedDelimiter\norm{\lVert}{\rVert}% for nice absolute value
                                            % bars
% For marginnotes in math floats
\newcommand{\filler}[1][10]%
{   \foreach \x in {1,...,#1}
    {   test
    }
}

\def\mathnote#1{%
  \tag*{\rlap{\hspace\marginparsep\smash{\parbox[t]{\marginparwidth}{%
  \footnotesize#1}}}}
}

%\renewcommand{\descriptionlabel}[1]{\hspace{\labelsep}\textsf{#1}}%

\begin{document}

\maketitle

\section{Exercise 1}

\emph{Show that (with $k$ being any integer):
  \begin{enumerate}[label=(\alph*)]
  \item $ \tan^{-1}(1 + i) = (-1/2)(\arctan 2 + \pi(1 + 2k)) + (1/4)i
    \ln 4$.
  \item $ \cosh^{-1}(-1) = \pi(2k + 1)i$.
  \item $ \tanh^{-1} 0 = \pi k i$.
  \end{enumerate}
}

\bigskip

\begin{enumerate}[label=(\alph*)]
\item
  \begin{align*}
    \arctan (1 + i)
    &= {1 \over 2} - \log\left( {i + z \over i - z}
      \right).\\
    w &= \arctan z\\
    z &= \tan w\\
    &= {1 \over i} \log \left( {i + (1 + i) \over i - (1 + i)}
      \right)\\
    &= {i \over 2} \log \left( { 1 + 2i \over -1 } \right)\\
    &= {i \over 2} \log(-1 - 2i)\\
    &= {i \over 2}(\ln\abs{-1 - 2i} + i \arg(-1 - 2i))\\
    &= {i \over 2}\left(\ln \sqrt{{(-1)}^2 + {(-2)}^2} + i(\Arg(-1 - 2i) +
      2 \pi k)\right), && k \in \mathbb{Z}\\
    &= {i \over 2}\left( \ln \sqrt 5 + i ((\arctan 2 + \pi) + 2\pi k) \right)\\
    &= {i \over 4} \ln 5 - {1 \over 2} (\arctan 2 + \pi ( 2k +
      1)).\mathnote{ I don't know either. It's what my notes say. \smiley{} }
  \end{align*}
\item
  \begin{align*}
    \arccosh(-1) &= \log\left(\sqrt{{(-1)}^2 - 1} - 1 \right)\\
                 &= \ln \abs{-1} + i \arg(-1)\\
                 &= \ln 1 + i(\Arg(-1) + 2 \pi k), && k \in \mathbb{Z}\\
                 &= i \pi + 2 \pi k i\\
                 &= \pi i (2k + 1).
  \end{align*}
\item
  \begin{align*}
    \arctanh 0 &= {1 \over 2} \log {1 + 0 \over 1 - 0}\\
               &= {1 \over 2} \log 1\\
               &= {1 \over 2}( \ln 1 + i \arg 1 )\\
               &= {i \over 2}(\Arg 1 + 2 \pi k), && k \in
                                                       \mathbb{Z}\\
               &= {i \over 2}2 \pi k\\
               &= \pi k i.
  \end{align*}
\end{enumerate}

\section{Exercise 2}

\emph{Explain why all values of $z \in \mathbb{C}$ satisfying $\cos z
  = \sqrt 2$ are given by $z = -2 \pi k \pm i \ln(\sqrt 2 + 1) $ for any
  integer $k$.}

\bigskip

Consider $\cos z = \sqrt{2}$. Then
\begin{align*}
  z &= \arccos \sqrt{2}\\
    &= {1 \over i} \log \left( \sqrt{2} + i {(1 - 2)}^{1/2}
      \right)\\
    &= {1 \over i} \log \left( \sqrt 2 \pm i \sqrt{-1} \right)\\
    &= {1 \over i} \log \left( \sqrt 2 \pm 1 \right)\\
    &= {1 \over i} \left( \ln {\abs*{\sqrt 2 \pm 1}} + i \left( \Arg \left(
      \sqrt 2 + 1 \right) + 2 \pi k \right) \right), && k \in
                                                        \mathbb{Z}\\
    &= {1 \over i} \left( \ln \left( \sqrt{2} + 1 \right) + i (0 + 2
      \pi k) \right)\\
    &= \pm i \ln \left( \sqrt{2} + 1 \right) + 2 \pi k.
\end{align*}

\section{Exercise 3}

\emph{Use the definition of a limit to prove the following:
  \begin{enumerate*}[label=(\alph*)]
  \item $\lim_{z \to z_0} \overline{z} = \overline{z_0}$,
  \item $\lim_{z \to 0} \overline{z}^2 / z = 0$.
  \end{enumerate*}
}

\bigskip

Definition 2 of Section 15 in Brown states that for each positive
number $\epsilon$, there is a positive number $\delta$ such that if
\[
\lim_{z \to z_0} f(z) = w_0
\]
then
\[
\abs{f(z) - w_0} < \epsilon\  \mathrm{whenever}\ 0 < \abs{z - z_0} < \delta.
\]

\begin{enumerate}[label=(\alph*)]
\item
  \begin{proof} Choose $\delta > \abs{z - z_0}$.\end{proof}
\item
  \begin{proof}
    \begin{align*}
      \abs*{{\overline{z}^2 \over z} - 0}
      &= \abs*{{(x - iy)}^2 \over x + iy}\\
      &= \abs*{x^2 -2ixy - y^2 \over x + iy}\\
      &= \dots
    \end{align*}
  \end{proof}
\end{enumerate}

\section{Exercise 4}

\emph{Show that the limit of the function $f(z) = (z /
  \overline{z}{)}^2$ as $z$ tends to $0$ does not exist. (Try
  approaching the origin on lines with varying slopes.)}

\bigskip

Consider the function\marginnote{Mostly from Brown 7th Ed.\ Solutions Manual.}
\begin{align*}
f(z) = {\left( z \over \overline{z} \right)}^2 = {\left( x + iy \over
    x - iy \right)}^2, && z \neq 0
\end{align*}
where $z = x + iy$. Observe that if $z = (x, 0)$, then
\[
f(z) = {\left( x + i0 \over x - i0 \right)}^2 = 1
\]
and if $z = (0, y)$ then
\[
f(z) = {\left( 0 + iy \over 0 - iy \right)}^2 = 1.
\]

But if $z = (x,x)$,
\[
f(z) = {\left( x + ix \over x - ix \right)}^2 = {\left( 1 + i \over 1
    - i \right)}^2 = -1.
\]

Thus the limit does not exist.

\section{Exercise 5}

\emph{Show that for any positive integer $n$, we have $\lim_{z \to
    z_0} z^n = z_0^n$.}

\emph{Quote any theorems that you use. For bonus points, prove this result
  exclusively with the definition of the limit.}

\bigskip

\proof Brown 16:2\marginnote{That's Section 16, Theorem 2.} states the
following:

\begin{quotation}
Suppose that
\[
\lim_{z \to z_0} f(z) = w_0\ \mathrm{and}\ \lim_{z \to z_0} F(z) = W_0.
\]

Then
\begin{align*}
\lim_{z \to z_0} (f(z) + F(z)) &= w_0 + W_0\\
\lim_{z \to z_0} (f(z)F(z)) &= w_0W_0,
\end{align*}
and if $W_0 \neq 0$,
\[
\lim_{z \to z_0} {f(z) \over F(z)} = {w_0 \over W_0}.
\]
\end{quotation}

Consider the base case, $n = 1$. Then
\begin{align*}
  \lim_{z \to z_0} z &= z_0.
\end{align*}

Assume for purposes of induction that the claim is true when $n =
k$. Then when $n = k + 1$,
\begin{align*}
  \lim_{z \to z_0} z^{k + 1} &= \lim_{z \to z_0} z^k \cdot
                               z\\
                             &= z_0^k \cdot z_0\mathnote{Apply Brown
                               16:2 with our assumption and base case.}\\
                             &= z_0^{k + 1}.
\end{align*}
\qed

\section{Exercise 6}

\emph{Use the theorems quoted in class (or in Section 17 of the
  textbook) to show that
  \begin{enumerate*}[label=(\alph*)]
  \item $\lim_{z \to \infty} 4z^2 / (z - 1{)}^2 = 4$,
  \item $\lim_{z \to 1} 1/(z - 1)^3 = \infty$, and
  \item $\lim_{z \to \infty} (z^2 + 1)/(z - 1) = \infty$.
  \end{enumerate*}}

\bigskip


\begin{enumerate}[label=(\alph*)]
\item
  \begin{align*}
    \lim_{z \to \infty} {4z^2 \over {(z - 1)}^2}
    &= \lim_{w \to 0} {4 -  {\left(1 \over w \right)}^2 \over {\left(
      {1 \over w} - 1 \right)}^2}\\
    &= \lim_{w \to 0} {4 \over {(1 - w)}^2}\\
    &= 4.
  \end{align*}
\item Note that $lim_{z \to z_0} f(z) = \infty$ if and only if $\lim_{z \to
  z_0} 1/f(z) = 0$. Then
  \begin{align*}
    \lim_{z \to 1} {1 \over {1 /{(z - 1)}^3 }}
    &= \lim_{z \to  1} {(z - 1)}^3 = 0.
  \end{align*}
\item Note that $\lim_{z \to \infty} f(z) = \infty$ if and only if
  $\lim_{z \to 0} 1 / f(1/z) = 0$. Then
\[
\lim_{z \to z_0} { {1 \over z} - 1 \over {\left( 1 \over z \right)}^2
  + 1} = \lim_{z \to 0} {z - z^2 \over 1 + z^2} = 0.
\]
\end{enumerate}

\section{Exercise 7}

\emph{Show that $f'(z)$ does not exist for any point $z$ when
  \begin{enumerate*}[label=(\alph*)]
  \item $f(z) = \Re z$,
  \item $f(z) = \Im z$, and
  \item $f(z) = \overline{z}^2 / z$ for $z \neq 0$ and $f(0) = 0$.
  \end{enumerate*}}

\bigskip

\begin{enumerate}[label=(\alph*)]
\item
  $f(z) = u(x,y) + iv(x,y)$ where $u(x,y) = x$, $v(x,y) =
  0$\marginnote{From \url{
      http://www.math.drexel.edu/~dmitryk/Teaching/MATH630-SPRING%2711/WA2_solutions.pdf
    }.}. Since $u_x = 1 \neq 0 = v_y$, the Cauchy--Riemann equations
  are not satisfied at any point. Therefore, $x$ is not differentiable
  at any point.
\item $f(z) = u(x,y) + iv(x,y)$ where $u(x,y) = y$, $v(x,y) =
  0$. Since $-v_x = 0 \neq 1 = u_y$, the Cauchy--Riemann equations are not
  satisfied at any point. Therefore, $f$ is not differentiable at any
  point.
\item In the case $f(0) = 0$,
  \begin{align*}
    f'(0)
    &= \lim_{\Delta z \to 0} {f(0 + \Delta z) - f(0) \over \Delta z}\\
    &= \lim_{\Delta z \to 0} { { {\left( \overline{\Delta z} \right)
      }^2 \over \Delta z} - 0 \over \Delta z}\\
    &= \lim_{\Delta z \to 0} { \left( \overline{\Delta z} \over \Delta
      z \right) }^2.
  \end{align*}
  Let $\Delta z = h+ ik$. Then
  \begin{align*}
    \lim_{\Delta z \to 0} { \left( \overline{\Delta z} \over \Delta
      z \right) }^2
    &= \lim_{\substack{h \to 0\\k \to 0}} {\left( h - ik \over h + ik
    \right)}^2.
  \end{align*}
  When $h = 0$, $f'(0) = 1$, but when $h = k$, $f'(0) = -1$. Then
  $f'(0)$ does not exist.

  In the case $z \neq 0$,
  \begin{align*}
    f'(z)
    &= \lim_{\Delta z \to 0} { { {\left( \overline{z + \Delta z}
      \right)}^2 \over z + \Delta z} - { { \left( \overline{z}
      \right)}^2 \over z} \over \Delta z}\\
    &= \lim_{\Delta z \to 0} {z { \left( \overline{z + \Delta z}
      \right)}^2 + {(\overline{z})}^2(z + \Delta z) \over z(\Delta z)(z +
      \Delta z)}\\
    &= \lim_{\Delta z \to 0} {\left( z{(\overline{z})}^2 +
      2z\overline{z} \overline{\Delta z} + z {\left( \overline{\Delta
      z} \right)}^2 \right) - \left(z {(\overline{z})}^2 + \Delta z
      {(\overline{z})}^2\right) \over z (\Delta z)(z + \Delta z)}
  \end{align*}

  When $\Delta x = 0$,
  \begin{IEEEeqnarray*}{rCl}
    \IEEEeqnarraymulticol{3}{l}{
      \lim_{\Delta y \to 0} {2 z \overline{z}(-i \Delta y) + z{(-i
        \Delta y)}^2 - \Delta y {(\overline{z})}^2 \over z \Delta y(z
      + \Delta y)}
    }\\ \quad
    & = & \lim_{\Delta y \to 0} {\Delta y(-2z\overline{z} - 2\Delta y
      - {(\overline{z})}^2) \over z (\Delta y)(z + \Delta y)}\\
    & = & {-2iz\overline{z} - {(\overline{z})}^2 \over z^2}.
  \end{IEEEeqnarray*}

  When $\Delta y = 0$,
  \begin{IEEEeqnarray*}{rCl}
    \lim_{\Delta x \to 0} { \Delta x(2 z \overline{z} + z \Delta x -
      \overline{z}^2) \over z \Delta x(z + \Delta x)}
    & = & {2z\overline{z} - {(\overline{z})}^2 \over z^2}.
  \end{IEEEeqnarray*}

  Therefore the limit does not exist.
\end{enumerate}

\end{document}

%%% Local Variables:
%%% mode: latex
%%% TeX-master: t
%%% End:
