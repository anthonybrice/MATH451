\documentclass{tufte-handout}

\title{Math 451: Homework 3}
\author{Anthony Brice}

\usepackage{graphicx} % allow embedded images
\setkeys{Gin}{width=\linewidth,totalheight=\textheight,keepaspectratio}
% \graphicspath{{graphics/}} % set of paths to search for images
\usepackage{amsmath, amsthm, amssymb}  % extended mathematics
\usepackage{booktabs} % book-quality tables
\usepackage{units}    % non-stacked fractions and better unit spacing
\usepackage{multicol} % multiple column layout facilities
\usepackage{multirow} % ???
\usepackage{lipsum}   % filler text
\usepackage{fancyvrb} % extended verbatim environments
  \fvset{fontsize=\normalsize}% default font size for fancy-verbatim
                              % environments
\usepackage[inline]{enumitem} % for fancy lists

\usepackage{newfloat}
\DeclareFloatingEnvironment[name=Listing]{mylisting}

\newenvironment{listingenv} {
  \begin{mylisting}
} {
  \end{mylisting}
}

\usepackage{subcaption}
\captionsetup{compatibility=false}

\usepackage[scaled]{berasans}
\usepackage[T1]{fontenc}
\usepackage{listings,xcolor}
%\lstloadlanguages{[5.2]Mathematica}
\lstset{language=Mathematica}

\lstset{basicstyle={\sffamily\footnotesize},
  numbers=left,
  numberstyle=\tiny\color{gray},
  numbersep=5pt,
  breaklines=true,
  captionpos={t},
  frame={lines},
  rulecolor=\color{black},
  framerule=0.5pt,
  columns=flexible,
  tabsize=2
}

% Standardize command font styles and environments
\newcommand{\doccmd}[1]{\texttt{\textbackslash#1}}% command name -- adds backslash automatically
\newcommand{\docopt}[1]{\ensuremath{\langle}\textrm{\textit{#1}}\ensuremath{\rangle}}% optional command argument
\newcommand{\docarg}[1]{\textrm{\textit{#1}}}% (required) command argument
\newcommand{\docenv}[1]{\textsf{#1}}% environment name
\newcommand{\docpkg}[1]{\texttt{#1}}% package name
\newcommand{\doccls}[1]{\texttt{#1}}% document class name
\newcommand{\docclsopt}[1]{\texttt{#1}}% document class option name
\newenvironment{docspec}{\begin{quote}\noindent}{\end{quote}}
% command specification environment

\newcommand{\e}[1]{\ensuremath{\times 10^{#1}}} % Macro for scientific
                                                % notation

\DeclareMathOperator{\Log}{Log}
\DeclareMathOperator{\Arg}{Arg}
\let\Im\relax
\DeclareMathOperator{\Im}{Im}
\let\Re\relax
\DeclareMathOperator{\Re}{Re}

% Use fancy symbols for footnotes
\usepackage{hyperref}
\usepackage{natbib}
\renewcommand{\thefootnote}{\fnsymbol{footnote}}%
\usepackage{perpage}
\MakePerPage{footnote}

\usepackage{manfnt} % ???

\usepackage{mathtools} % for \DeclarePairedDelimiter

\DeclarePairedDelimiter\abs{\lvert}{\rvert}%
\DeclarePairedDelimiter\norm{\lVert}{\rVert}% for nice absolute value
                                            % bars

%\renewcommand{\descriptionlabel}[1]{\hspace{\labelsep}\textsf{#1}}%

\begin{document}

\maketitle

\section{Exercise 1}

\emph{Show that}
\begin{enumerate*}[label=\emph{(\alph*)}]
\item \emph{$\Log (-ei) = 1 - i \pi / 2$, and}
\item \emph{$\Log(1 - i) = (1/2) \ln 2 - i \pi / 4$.}
\end{enumerate*}

\bigskip

We repeat some useful definitions related to the principal
value\footnotemark{} for posterity: \footnotetext{Despite the claim of
  many students convinced only their \emph{principal} is their
  \emph{pal}, \emph{principle} is never an adjective.}
\begin{gather*}
\log z = \ln \abs{z} + i \arg z.\\
\Log z = \ln\abs{z} + i \Arg z.
\end{gather*}
Note that $\arg z = \Arg z + 2 \pi i k$ for any $k \in z$.

\begin{enumerate}[label=(\alph*)]
\item
Consider $\Log w$ defines the complex number that solves $e^z = w$
such that $-\pi < \Im z < \pi$.
\begin{align*}
  e^z &= -ei\\
  e^{x + iy} &= e \cdot e^{-i \pi / 2}\\
      &= e^1 \cdot e^{-i \pi / 2}\\
      &= e^{1 - i \pi / 2}\\
  x + i y &= 1 - i {\pi \over 2}.
\end{align*}

\item
\begin{align*}
  \Log(1 - i) &= \ln\abs{1 - i} + i \Arg(1 - i)\\
              &= \ln\abs{1 - i} - i {\pi \over 4}\\
              &= \ln \sqrt{2} - i {\pi \over 4}\\
              &= {1 \over 2} \ln 2 - i {\pi \over 4}.
\end{align*}
\end{enumerate}

\section{Exercise 2}

\emph{Show that}
\begin{enumerate*}[label=\emph{(\alph*)}]
\item \emph{$\log e = 1 + 2 n \pi i$,}
\item \emph{$\log i = (2 \pi + 1/2) \pi i$, and}
\item \emph{$\log(-1 + i \sqrt 3) = \ln 2 + 2(n + 1/3)\pi i$ for any
    integer $n$.}
\end{enumerate*}

\bigskip

\begin{enumerate}[label=(\alph*)]
\item
  \begin{align*}
    \log e &= \ln e + i (0 + 2 \pi n) && n \in \mathbb{Z}\\
           &= 1 + 2 i \pi n.
  \end{align*}
\item
  \begin{align*}
    \log i &= \ln 1 + i \left( {\pi \over 2} + 2 \pi n \right) && n
                                                                  \in
                                                                  \mathbb{Z}\\
    &= i \pi \left(2n + {1 \over 2} \right).
  \end{align*}
\item
  \begin{align*}
    \log(-1 + i \sqrt 3) &= \ln 2 + i \left({2 \pi \over 3} + 2 \pi n
                           \right) && n \in \mathbb{Z}\\
                         &= \ln 2 + 2 i \pi \left( n + {1 \over 3}\right).
  \end{align*}
\end{enumerate}

\section{Exercise 3}

\emph{Show that $\Log(1 + i{)}^2 = 2 \Log(1 + i)$, but
  $\Log(-1 + i{)}^2 \neq 2 \Log(-1 + i)$.}

\bigskip

\begin{align*}
  \Log(1 + i{)}^2 &= \Log(2i)\\
                  &= \ln 2 + i {\pi \over 2}
\end{align*}
and
\begin{align*}
  2\Log(1 + i) &= 2 \left(\ln \sqrt 2 + i {\pi \over 4} \right)\\
               &= \ln 2 + i {\pi \over 2}.
\end{align*}

But
\begin{align*}
  \Log(-1 + i{)}^2 &= \Log(-2i)\\
                   &= \ln 2 - i {\pi \over 2}
\end{align*}
and
\begin{align*}
  2 \Log(-1 + i) &= 2 \left( \ln \sqrt 2 + i {3 \pi \over 4} \right)\\
                 &= \ln 2 + i {3 \pi \over 2}.
\end{align*}

\section{Exercise 4}

\emph{Show that ${(1 + i)}^i = e^{-(\pi/4 + 2 \pi k)} \cdot e^{i \ln 2
  / 2}$ and ${(-1)}^{1/\pi} = e^{(2k + 1)i}$ for any integer $k$.}

\bigskip

\begin{align*}
  (1 + i{)}^i &= \exp(i \log (1 + i))\\
              &= \exp(i(\ln\abs{1 + i} + i \arg(1 + i)))\\
              &= \exp(i(\ln \sqrt 2 + i(\Arg(1 + i) + 2 \pi k)))
              && k \in \mathbb{Z}\\
              &= \exp\left(i \left( {1 \over 2} \ln 2 + i \left({\pi \over
                4} + 2 \pi k \right) \right) \right)\\
              &= \exp \left( {1 \over 2} i \ln 2 - {\pi \over 4} - 2
                \pi k \right)\\
              &= e^{-(\pi/4 + 2 \pi k)} \cdot e^{i \ln 2 / 2}.
\end{align*}

\begin{align*}
  (-1{)}^{1/\pi} &= \exp \left( {1 \over \pi} \log (-1) \right)\\
                 &= \exp \left( {1 \over \pi} ( \ln \abs{-1} + i
                   \arg(-1) ) \right)\\
                 &= \exp \left( {i \over \pi}(\Arg(-1) + 2 \pi k)
                   \right)
                 && k \in \mathbb{Z}\\
                 &= \exp \left( {i \over \pi}(\pi + 2 \pi k) \right)\\
                 &= \exp(i + 2 i k)\\
                 &= e^{i(2k + 1)}.
\end{align*}

\section{Exercise 5}

\emph{Is $1$ raised to any power always equal to $1$? Explain.}

\bigskip

$1$ raised to any power is not always equal to $1$. Consider the
$n$-th root of unity.

\section{Exercise 6}

\emph{Give an example to show that the principal value of
  ${(z w)}^\alpha$ is not always equal to the product of the principal
  value of $z^\alpha w^\alpha$ for some $z,w,\alpha \in \mathbb{C}$.}

\bigskip

Choose $z = e/2$, $w = -1 - i \sqrt 3$, and $\alpha = 3 \pi i$.

\section{Exercise 7}

\emph{Find the principal values of ${[(e/2)(-1 - i \sqrt 3)]}^{3 \pi
    i}$ and ${(1 - i)}^{4i}$.}

\emph{\emph{Answers:} $-\exp(2 \pi^2)$ and $e^\pi [\cos(2 \ln 2) + i
  \sin (2 \ln 2)]$.}

\bigskip

The principal value of ${[(e/2)(-1 - i \sqrt 3)]}^{3 \pi i}$ is
\begin{align*}
  e^{3 \pi i \Log\left( (e/2) (-1 - i \sqrt 3) \right)}
  &= \exp\left( 3 \pi i \left[ \ln\abs*{{e \over 2}(-1 - i \sqrt 3)} +
    i \Arg\left({e \over 2}(-1 - i \sqrt 3) \right) \right] \right) \\
  &= \exp\left(3 \pi i \left[ \ln \left( {e \over 2} \cdot 2 \right) +
    i \cdot {-2 \pi \over 3} \right] \right)\\
  &= \exp\left(3 \pi i + 2 \pi^2 \right)\\
  &= e^{3 \pi i} \cdot e^{2 \pi^2}\\
  &= -e^{2 \pi^2}.
\end{align*}

The principal value of $(1 - i{)}^{4i}$ is
\begin{align*}
  \exp( 4i \Log(1 - i))
  &= \exp( 4i (\ln \abs{1 - i} + i \Arg(1 - i)))\\
  &= \exp\left( 4 i \left(\ln \sqrt 2 - i {\pi \over 4} \right)
    \right)\\
  &= \exp \left( 4 i \left( {1 \over 2} \ln 2 - i {\pi \over 4}
    \right) \right)\\
  &= \exp( 2 i \ln 2 + \pi)\\
  &= e^\pi \cdot e^{2i \ln 2}\\
  &= e^\pi [ \cos(2 \ln 2) + i \sin(2 \ln 2)].
\end{align*}

\section{Exercise 8}

\emph{Show that if $z \neq 0$ and $a \in \mathbb{R}$, then $\abs{z^a}
  = \exp(a \ln \abs{z}) = \abs{z}^a$ where the principal value of
  $\abs{z}^a$ is to be taken.}

\bigskip

\begin{align*}
  \abs{z^a} &= \abs{\exp(a \log z)}\\
            &= \abs{\exp(a\left( \ln \abs{z} + i \arg z \right))}\\
            &= \abs{\exp(a \ln \abs{z}) \exp(a i \arg z)}\\
            &= \abs{\exp(a \ln \abs{z})} \cdot \abs{\exp(a i \arg
              z)}\\
            &= \exp(a \ln \abs{z}) \cdot 1\\
            &= \abs{z}^a.
\end{align*}

\section{Exercise 9}

\emph{By using Euler's identity, show that for any $\theta \in
  \mathbb{R}$, we have}%
\[
\cos \theta = {1 \over 2} \left( e^{i \theta} + e^{-i \theta} \right)
\]
\emph{and}%
\[
\sin \theta = {1 \over 2i} \left( e^{i \theta} - e^{-i \theta} \right).
\]

\bigskip

Consider
\begin{align*}
  e^{i \theta}
  &= \cos \theta + i \sin \theta\\
  e^{-i \theta}
  &= \cos \theta - i \sin \theta.
\end{align*}
Then sum and solve for $\cos \theta$ or subtract and solve for $\sin
\theta$.

\section{Exercise 10}

\emph{Define the hyperbolic function $\cosh z = (1/2)(e^z + e^{-z})$
  and $\sinh z = (1/2)(e^z - e^{-z})$. Prove that $\cosh (i z) = \cos
  z$ and $-i \sinh(i z) = \sin z$.}

\bigskip

\begin{align*}
  \cosh(i z) &= {1 \over 2} \left( e^{i z} + e^{-i z} \right)\\
             &= \cos z.
\end{align*}
\[ \sin(x) \]
\begin{align*}
  i \sinh(i z) &= -i \cdot {e^{i z} - e^{-i z} \over 2}\\
               &= {e^{i z} - e^{-i z} \over 2i}\\
               &= \sin z.
\end{align*}

\end{document}

%%% Local Variables:
%%% mode: latex
%%% TeX-master: t
%%% End:
